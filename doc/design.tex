\section{Modelling the Input Data}

Before the whole algorithm can be modelled mathematically, we must first model 
the input data as set of variables.
\begin{itemize}
	\item Let \( T = \{ t_1 , \ldots , t_n \} \) be the set of teachers where:
	\begin{itemize}
		\item a teacher \( t_i = \{ M_{t_i}, H_{t_i} \} \) where:
		\begin{itemize} 
			\item \( H_{t_i} = \{ h_{t_{i_1}} , \ldots , h_{t_{i_n}} \} \) is 
				the set of the teacher's preferred timeslots, and
			\item \( M_{t_i} = \{ m_{t_{i_1}} , \ldots , m_{t_{i_n}} \} \) is 
				the set of modules that the teacher teaches.
		\end{itemize}
	\end{itemize}
	\item Let \( R = \{ r_1 , \ldots , t_n \} \) be the set of rooms where:
	\begin{itemize}
		\item a room \( r_i = \{ y_{r_i} , c_{r_i} \} \) where:
		\begin{itemize}
			\item \( y_{r_i} \) is the type of room, which takes the value 
				\( 1 \) if the room is a lecture theatre, or \( 2 \) if the room
				is a lab, and
			\item \( c_{r_i} \) is the maximum capacity of the room.
		\end{itemize}
	\end{itemize}
	\item Let \( S = \{ s_1 , \ldots , s_n \} \) be the set of student groups
		where:
	\begin{itemize}
		\item a student group \( s_i = \{ M_{s_i} , z_{s_i} \} \) where:
		\begin{itemize}
			\item \( M_{s_i} = \{ m_{s_{i_1}} , \ldots \, m_{s_{i_n}} \} \) is 
				the set of modules taken by the student group \( s_i \), and
			\item \( z_{s_i} \) is the number of students in \( s_i \).
		\end{itemize} 
	\end{itemize}
	\item Let \( M = \{ m_1, \ldots, m_n \} \) be the set of modules where:
	\begin{itemize}
		\item a module \( m_i = \{ p_{m_i}, q_{m_i} \} \) where:
		\begin{itemize}
			\item \( p_{m_i} \) is the number of lecture hours of the module 
				\( m_i \), and 
			\item \( q_{m_i} \) is the number of lab hours of the module 
				\( m_i \).
		\end{itemize}
	\end{itemize}
	\item Let \( H = \{ h_1 , \ldots , h_n \} \) be the set of time slots.
\end{itemize}
Further to this, we can also describe a set of teaching sessions \( X \) where 
each session \( x_{t_{\alpha},r_{\beta},s_{\gamma},m_{\delta},h_{\epsilon}} \) 
is a session taught by teacher \( t_{\alpha} \), in room \( r_{\beta} \), with 
student group \( s_{\gamma} \), about module \( m_{\delta} \), during time slot
\( h_{\epsilon} \).

\section{Constraints}

We next need to represent the constraints. We take the requirements from \S 2.1,
but add two more constraints (2 and 3).
\begin{enumerate}
	\item Every module has all its hours timetabled.
	\item Each session is taught by the correct teacher for the module.
	\item Each session has the correct student group attending it.
	\item A room is being used for at most one learning session in a time slot.
	\item A student has at most one learning session in a time slot.
	\item A teacher has at most one learning session in a time slot.
	\item A session is in the correct type of room, for example, a lecture is in 
		a lecture theatre, and a lab session is in a lab.
	\item The number of students in the room must not exceed the capacity of the
		room.
	\item The size of the room is optimal for the number of students, meaning 
		that the size of the room is as small as possible.
	\item Teachers can make preferences for when they are teaching.
\end{enumerate}
We will now formulate these constraints below.

\subsection{Every module has all its hours timetabled}

We can say that the number of sessions scheduled is equal to the sum of the 
number of lecture sessions and the number of lab sessions for all the modules, 
or:
\begin{equation*}
	\lvert X \rvert = \sum_{i=1}^{\lvert M \rvert} (p_{m_i} + q_{m_i})
\end{equation*}
However, this is not sufficient to prove that the correct number of sessions per
module has been scheduled, for instance one module might be missing a session 
and another module might have an extra session scheduled, and the above 
statement would still hold. To further prove this, we also have constraint 7.

\subsection{Each session is taught by the correct teacher for the module}

This can be reworded to: for every session, the module taught in that session is
in the set of modules taught by the teacher, or:
\begin{equation*}
	\forall x_{t_{\alpha},r_{\beta},s_{\gamma},m_{\delta},h_{\epsilon}} \in X,
	\: m_{\delta} \in M_{t_{\alpha}}
\end{equation*}

\subsection{Each session has the correct student group attending it}

Very similarly to the previous constraint, this constraint can be reworded to: 
for every session, the module taught in that session is in the set of modules 
taken by the student group, or:
\begin{equation*}
	\forall x_{t_{\alpha},r_{\beta},s_{\gamma},m_{\delta},h_{\epsilon}} \in X,
	\: m_{\delta} \in M_{s_{\gamma}}
\end{equation*}

\subsection{A room is being used for at most one learning session in a time 
	slot}

This can be rephrased as: for each room, there is at most one of each time slot 
scheduled for that room.
\begin{equation*}
	\forall r_i \in R, \: \exists X_{r_i} \subseteq X \text{ where } X_{r_i} = 
	\{ x_{t_{\alpha},r_{\beta},s_{\gamma},m_{\delta},h_{\epsilon}} \: | \: 
	r_{\beta} = r_i\}
\end{equation*}
This creates subsets \( X_{r_i} \) which are all the learning sessions scheduled
to the room \( r_i \). Then:
\begin{equation*}
	\forall h_j \in H\, \: \exists X_{r_i,h_j} \subseteq X_{r_i} \text{ where }
	X_{r_i,h_j} = \{ x_{t_{\alpha}, r_i, s_{\gamma}, m_{\delta}, h_{\epsilon}} 
	\: | \: h_{\epsilon} = h_j \}
\end{equation*}
This creates subsets \( X_{r_i,h_j} \) which are all the learning session 
scheduled in time slot \( h_j \) for the room \( r_i \). We then require the 
condition
\begin{equation*}
	\lvert X_{r_i,h_j} \rvert = 1 \text{ or } \lvert X_{r_i,h_j} \rvert = 0
\end{equation*}
to hold \( \forall X_{r_i,h_j} \).

\subsection{A student has at most one learning session in a time slot}

This very similar to constraint 4.
\begin{equation*}
	\forall s_i \in S, \: \exists X_{s_i} \subseteq X \text{ where } X_{s_i} = 
	\{ x_{t_{\alpha},r_{\beta},s_{\gamma},m_{\delta},h_{\epsilon}} \: | \: 
	s_{\gamma} = s_i\}
\end{equation*}
\begin{equation*}
	\forall h_j \in H\, \: \exists X_{s_i,h_j} \subseteq X_{s_i} \text{ where }
	X_{s_i,h_j} = \{ x_{t_{\alpha}, r_{\beta}, s_i, m_{\delta}, h_{\epsilon}} 
	\: | \: h_{\epsilon} = h_j \}
\end{equation*}
It is then required that
\begin{equation*}
	\lvert X_{s_i,h_j} \rvert = 1 \text{ or } \lvert X_{s_i,h_j} \rvert = 0, \: 
	\forall X_{s_i,h_j}.
\end{equation*}

\subsection{A teacher has at most one learning session in a time slot}

This very similar to constraints 4 and 5.
\begin{equation*}
	\forall t_i \in T, \: \exists X_{t_i} \subseteq X \text{ where } X_{t_i} = 
	\{ x_{t_{\alpha},r_{\beta},s_{\gamma},m_{\delta},h_{\epsilon}} \: | \: 
	t_{\alpha} = t_i\}
\end{equation*}
\begin{equation*}
	\forall h_j \in H\, \: \exists X_{t_i,h_j} \subseteq X_{t_i} \text{ where }
	X_{t_i,h_j} = \{ x_{t_i, r_{\beta}, s_{\gamma}, m_{\delta}, h_{\epsilon}} 
	\: | \: h_{\epsilon} = h_j \}
\end{equation*}
It is then required that
\begin{equation*}
	\lvert X_{t_i,h_j} \rvert = 1 \text{ or } \lvert X_{t_i,h_j} \rvert = 0, \: 
	\forall X_{t_i,h_j}.
\end{equation*}

\subsection{A session is in the correct type of room}

This can be thought of as: for each module, the number of sessions in a lecture 
room is equal to the number of lecture sessions for that module. The same 
applies for lab sessions.

To begin with (recall that \( y_{r_i} = 1 \) if \( r_i \) is a lecture room):
\begin{equation*}
	\exists X_{lecture} \subseteq X \text{ where } X_{lecture} = \{ 
	x_{t_{\alpha}, r_{\beta},s_{\gamma},m_{\delta},h_{\epsilon}} \: | \: \forall 
	r_{\beta}, \: y_{r_{\beta}} = 1\}
\end{equation*}
This gives us \( X_{lecture} \), a set that contains all the sessions in a 
lecture room. Then:
\begin{equation*}
	\forall m_i \in M, \: \exists X_{lecture,m_i} \subseteq X_{lecture} 
	\text{ where } X_{lecture,m_i} = \{ t_{\alpha},r_{\beta},s_{\gamma},
	m_{\delta},h_{\epsilon} \: | \: m_{\delta} = m_i\}
\end{equation*}
Finally, it is required that
\begin{equation*}
	\forall X_{lecture,m_i}, \: \lvert X_{lecture,m_i} \rvert = p_{m_i}.
\end{equation*}

Similarly, for labs:
\begin{equation*}
	\exists X_{lab} \subseteq X \text{ where } X_{lab} = \{ x_{t_{\alpha}, 
	r_{\beta},s_{\gamma},m_{\delta},h_{\epsilon}} \: | \: \forall r_{\beta}, \: 
	y_{r_{\beta}} = 2\}
\end{equation*}
\begin{equation*}
	\forall m_i \in M, \: \exists X_{lab,m_i} \subseteq X_{lab} \text{ where } 
	X_{lab,m_i} = \{ t_{\alpha},r_{\beta},s_{\gamma},m_{\delta},h_{\epsilon} \:
	| \: m_{\delta} = m_i\}
\end{equation*}
and it is required that
\begin{equation*}
	\forall X_{lab,m_i} \lvert X_{lab,m_i} \rvert = p_{m_i}.
\end{equation*}

As teaching sessions are either labs or lectures, it can be implied that \( 
\lvert X_{lecture} \rvert + \lvert X_{lab} \rvert = \lvert X \rvert \).

Because of the above, there are the correct number of lectures and labs for 
each module, and the conditions are now sufficient for constraint 1 to hold. 

\subsection{The number of students in the room must not exceed the capacity of 
	the room}

This can be reworded to: for every session, the size of the student group must 
be less than or equal to the capacity of the room, or:
\begin{equation*}
	\forall x_{t_{\alpha},r_{\beta},s_{\gamma},m_{\delta},h_{\epsilon}} \in X, 
	\: z_{s_{\gamma}} \leq c_{r_{\beta}}
\end{equation*}

\subsection{The size of the room is optimal for the number of students}

This means, for every session \( x_i \in X \) in room \( r_i \), if there 
exists a room \( r_j \in R \) where \( c_{r_j} < c_{r_i} \), change \( 
r_i \) to \( r_j \), only if no other constraints are violated. Or, \( r_i = 
\min \{ r_j \: | \: c_{r_j} \geq c_{r_i} \} \in R \).

\subsection{Teachers can make preferences for when they are teaching}

We first define a set of sessions for each teacher:
\begin{equation*}
	\forall t_i \in T, \: \exists X_{t_i} \subseteq X \text{ where } X_{t_i} = 
	\{ x_{t_{\alpha},r_{\beta},s_{\gamma},m_{\delta},h_{\epsilon}} \: | \: 
	t_{\alpha} = t_i\}
\end{equation*}
We then define a set containing just the time slots for each \( X_{t_i} \):
\begin{equation*}
	\forall X_{t_i}, \: \exists H_{xt_i} \text{ where } H_{xt_i} = \{ h_j \: | 
	\: h_j = h_{\epsilon} \forall x_{t_i,r_{\beta},s_{\gamma},m_{\delta},
	h_{\epsilon}} \in X_{t_i}\}
\end{equation*}
Then, for the constraint to be satisfied, \( H_{xt_i} \subseteq H_{t_i} \) must
hold \( \forall t_i \in T \), or alternatively, the constraint is violated if
\( \exists H_{xt_i} \ni h_j \notin H_{t_j} \).

\section{Timetable Encoding}

A solution will be encoded as a set of 5-tuples, where each tuple is 
\{time slot, room, student group, module, teacher\}, where a solution represents
an entire timetable.

For the genetic algorithm, the population is a set of solutions.

\newpage

\section{Genetic Algorithm Pseudocode}

\begin{listing}[!ht]
	\inputminted[linenos, fontsize=\footnotesize]{text}{code/ga-pseudocode.txt}
	\caption{Pseudocode for the genetic algorithm}
	\label{listing:genetic-algorithm}
\end{listing}

The settings data contains \verb|populationSize| and \verb|mutationChance|, 
which are parameters for the genetic algorithm the user may wish to change.
\verb|populationSize| is quite self-explanatory, simply the number of 
individuals in the population.
\verb|mutationChance| is the reciprocal of the actual chance of a mutation 
occurring, for example, if the chance of a mutation occurring is 
\( \frac{1}{1000} \), then \verb|mutationChance| needs to be 1000.

Note that the \verb|CheckFitness| function is called both inside the while loop
and before it.
This is so that the initial population generated can be checked for a valid 
solution, removing the need to proceed with the rest of the algorithm.
Hence, the population's fitness needs to be calculated again inside the while
loop, after the mutation phase.

\subsection{Initial Population}

\begin{listing}[!ht]
	\inputminted[linenos, fontsize=\footnotesize]{text}{code/population.txt}
	\caption{Pseudocode for generating the initial population}
	\label{listing:population}
\end{listing}

The initial population is generated within two for loops, as the population
consists of many solutions, each of which consist of many sessions.
For each session, the time slot and room are chosen randomly, but a session's 
module, student group and teacher is predetermined.
This is because when the configuration data is imported, a module is imported 
with its student group and teacher, as a triple.
Then each triple is stored in the list \verb|sessions|.
This means that a session automatically has the correct student group and 
teacher for the module, therefore reducing the difficulty of the problem.
The session is stored \( n \) times in \verb|sessions|, where \( n \) is the 
number of lab and lecture sessions of that module, so that the correct number 
of teaching sessions are created.

\newpage

\subsection{Fitness Function}

\begin{listing}[!ht]
	\inputminted[linenos, fontsize=\footnotesize]{text}{code/fitness.txt}
	\caption{Pseudocode for the fitness function}
	\label{listing:fitness}
\end{listing}

The fitness function is calculated by totalling the number of constraint 
violations, then raising it to the power of -1, so a higher fitness function is 
better.
However, solutions with no constraint violations have a fitness function of 0.

Violations are worked out in different ways depending on which constraint is
violated.
To check for clashes for student groups, rooms, and teachers, the sessions in a 
solution are sorted by time slot, so it can be checked if two consecutive 
sessions in the solution list share the same time slot.
If the two sessions do share the same time slot, then it is checked if the rooms
are the same, and then if they are, constraint 4 has been violated, and the 
count of violations is increased --- the same applies to student groups and 
teachers.

To check a session against the teacher's time preferences, the session's time 
slot is checked against a list of that teacher's preferred time slots.
If the session's time slot does not appear on the list, then constraint 10 has 
been violated and the number of and the count of violation is increased.

\newpage

\subsection{Selection}

\begin{listing}[!ht]
	\inputminted[linenos, fontsize=\footnotesize]{text}{code/select-parents.txt}
	\caption{Pseudocode for the selection phase}
	\label{listing:select-parents}
\end{listing}

\begin{listing}[!ht]
	\inputminted[linenos, fontsize=\footnotesize]{text}{code/choose-parent.txt}
	\caption{Pseudocode for a subroutine of the selection phase}
	\label{listing:choose-parent}
\end{listing}

Firstly, the fitness values of the population are normalised so that they add 
up to one. The selection of a parent can be modelled like a spinner wheel, where
the fitter the solution, the larger its `slice' on the spinner wheel, so it is 
more likely to be chosen.
The function \verb|NormaliseValues| returns a list of limits, which are the 
intermediate points of the cumulative sum of the normalised fitness values.
As an example, if the population has normalised fitness values of 
\( 0.25,0.5,0.25 \), then the returned \verb|normalisedLimits| will be 
\( [0,0.25,0.75,1] \).
Referring back to the spinner wheel analogy, these refer to the `edges' where 
the `slices' meet.

To select a parent, a random float between 0 and 1 is selected, which represents
where the spinner lands.
To continue the previous example, if the random float \verb|choice| is 0.52, 
then the second solution will be chosen, as \( 0.25 < 0.52 < 0.75 \).
Once this is selected, a binary search is performed to find the index of the 
chosen solution (\verb|parent|). After the first parent has been selected, its 
fitness value is removed from the list and the \verb|normalisedLimits| are 
recalculated and \verb|ChooseParent| is run again, to determine the second 
parent.

\newpage

\subsection{Crossover}

\begin{listing}[!ht]
	\inputminted[linenos, fontsize=\footnotesize]{text}{code/crossover.txt}
	\caption{Pseudocode for the crossover phase}
	\label{listing:crossover}
\end{listing}

Recall that to generate offspring, a locus in the chromosome is randomly chosen,
and the before and after parts of the chromosome are swapped between the two 
parents to produce two offspring.
So the first task of the crossover phase is determining where the locus will be 
in the chromosome.
As the chromosome in this instance is a solution, which is modelled as set of 
sets, we first need to determine which session contains the split --- which is 
indexed with \verb|outerLocus|.
The \verb|innerLocus| determines whether the locus is before the time slot
(\verb|innerLocus| \( = 0\)), before the room (\verb|innerLocus| \( = 1 \)), or
before the student group (\verb|innerLocus| \(=2\)). There is no split between
the student group and the module, or the module and the teacher, because that 
part of the solution is already known to be correct, so there is no point in
changing it.

The way that the crossover phase typically works is that the locus is in between
alleles, i.e.\ between characters in a string. 
However, due to how the data has been encoded, splitting between characters 
would create e.g.\ rooms that do not exist, which is counterproductive. 

\newpage

\subsection{Mutation}

\begin{listing}[!ht]
	\inputminted[linenos, fontsize=\footnotesize]{text}{code/mutation.txt}
	\caption{Pseudocode for the mutation phase}
	\label{listing:mutation}
\end{listing}

There is a chance of mutation for individual genes, rather than for the 
individual as a whole, meaning an individual could have multiple mutations.
Like with the crossover phase, the student group, module and teacher are not 
mutated separately, for the same reason.