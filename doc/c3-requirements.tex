\section{Necessary Requirements}

These are requirements that are specified by the project briefing, and it is
ideal that they are all achieved.
\begin{enumerate}
	\item The required information for the timetable can be entered by the user.
	\item A correct timetable is generated, where:
	\begin{enumerate}
		\item Every module has all its hours timetabled.
		\item A room is being used for at most one learning session in a time 
			slot.
		\item A student has at most one learning session in a time slot.
		\item A teacher has at most one learning session in a time slot.
		\item A session is in the correct type of room, for example, a lecture
			is in a lecture theatre, and a lab session is in a lab.
		\item The number of students in the room must not exceed the capacity of
			the room.
		\item The size of the room is optimal for the number of students,
			meaning that the size of the room is as small as possible.
		\item Teachers can make preferences for when they are teaching.
	\end{enumerate}
	\item The generated timetable is returned to the user in a way that they can
		understand.
\end{enumerate}

\section{Aspirational Requirements}

These are requirements that could be achieved, but are not essential for
completion.
\begin{enumerate}
	\item Use a graphical user interface.
	\item Allow the user to change the parameters of the genetic algorithm.
	\item Return the final timetable as multiple timetables individualised to
		each teacher, student group, room, etc. The timetables should be in a
		very legible format, such as in a traditional tabular timetable format.
\end{enumerate}