\section{Introduction to the Timetabling Problem}

The timetabling problem is a constraint satisfying problem and it is known to
be NP-Complete.
The aim is to create a university timetable where several constraints are met.
These constraints are either hard constraints or soft constraints.
For a solution to be valid, all the hard constraints must be met.
Not all (or in fact, any) of the soft constraints need to be met for the
solution to be valid, however it is preferable for as many soft constraints to
be met as possible.
Examples of possible hard constraints include:

\begin{itemize}
	\item At most only one session (i.e., a lecture or lab) is happening in a 
		specific room in a specific period
	\item A student can only be attending at most one session in a specific 
		period
	\item A teacher (e.g., a lecturer or lab helper) can only be attending at 
		most one session in a specific period
	\item The size of a student group cannot exceed the capacity of the room
	\item The room must be appropriate for the type of session, e.g., a lecture 
		must be in a lecture theatre, and a lab session must be in a computer 
		lab
	\item Part time teachers can only be assigned certain time slots, e.g., they
		may not work on Tuesdays, so sessions they teach cannot be scheduled
		for time slots on Tuesday.
\end{itemize}

Possible soft constraints include:

\begin{itemize}
	\item Students do not have more than two consecutive hours scheduled
	\item The capacity of a room is well suited to the size of the student
		group, to make an efficient use of space e.g., a group of twenty
		students are not going to be in a room with a capacity of two hundred
	\item If a student or teacher has one session immediately after another,
		then the respective rooms are relatively close to each other
\end{itemize}

A solution is invalid if at least of one the hard constraints are met.
For example, if a student is scheduled to be in two different sessions at the
same time --- this is known as a clash.

\section{Methods}

\subsection{Genetic Algorithms}

Genetic algorithms (GAs) make up a group of search metaheuristics, inspired by
Darwin's theory of evolution~\cite{ga_book}.
Here, the fittest members of a population survive and produce offspring, which
inherit the characteristics of the parents.
It is also possible for the offspring to have small mutations within their
genetic code, which may or may not be beneficial towards the population's
survival.
This theory can be applied to search problems.
In this case, the population represents the search space, which is a collection 
of candidate solutions to a problem, and the population of solutions evolves as
the algorithm searches for a desired solution.

There does not exist a rigorous definition of GAs, but most methods use these
five phases:

\begin{enumerate}
	\item Initial population --- populations of chromosomes
	\item Fitness function
	\item Selection --- according to fitness
	\item Crossover --- to produce new offspring
	\item Mutation --- random mutation of new offspring
\end{enumerate}

The initial \textbf{population} is a set of \textbf{individuals}, where each
individual represents a candidate solution to the problem.
These solutions will almost definitely not satisfy the problem, as they are
randomly generated, but that is not an issue.

An individual (and hence that candidate solution), is defined by its
\textbf{chromosome}.
The chromosome is often encoded as a string of binary characters; however, any
alphabet can be used.
These characters are called \textbf{alleles}, and a single character or group of
adjacent characters that encode a particular element of the candidate solution
are known as a \textbf{gene}.

Using the example of the university timetabling problem, several alleles may be
used to encode a room, but together they are one gene.

The \textbf{fitness} function measures how well a candidate solution solves the
problem.
For example, in the instance of the university timetabling problem, the fitness 
of the solution could be measured by the number of times in the solution that a
hard constraint is not met.
This means that a solution with a score of 0 is a valid solution, as there are
instances of a hard constraint not being met.

Once the fitness function has been used to calculate the fitness of each
individual, the fittest individuals are chosen for reproduction in the
\textbf{selection} phase.
There are several ways to select individuals to use for producing offspring.
One way is to simply choose the two individuals with the best fitness.
Another way is to use roulette-wheel sampling, where any individual could be
chosen for producing offspring, but fitter individuals are more likely to be
chosen.
This second method introduces more variation into the offspring, to reduce the
chance of convergence onto a local maximum.

The \textbf{crossover} phase, or reproduction phase, is where genetic material
is exchanged between two parents.
The crossover operator randomly chooses a locus (position) in a chromosome,
in between alleles.
The subsequent before and after parts of the chromosome are swapped between the
two parents to produce two offspring.
This is repeated multiple times to produce a population of the same size as the
initial population.
Whether this is by two parents reproducing multiple times using different loci
or not, depends on the method used in the selection phase.

After reproduction, \textbf{mutations} can be introduced into the offspring. 
For example, if the chromosomes are encoded by binary strings, then some bits 
are randomly flipped.
However, there is a very small chance of this occurring at each bit, a suggested
probability is 0.1\%.
By introducing mutations into the offspring, the likelihood of reaching a local
maximum is reduced.

Once there is a new population made up of the offspring, the process repeats 
until a solution is found.
Each repetition is called a \textbf{generation}, and the entire set of
generations is known as a \textbf{run}.

A version of a genetic algorithm was implemented by Perzina 
(2006)~\cite{ga_example} and was applied to a timetabling problem with real 
world data.
Timetables were generated for 1807 students, but for how courses are organised 
at this university, there are no groups of students with the same timetable, and
it is unlikely that two students would even have the same timetable.
Because of this, the author anticipated that it would be too difficult to
generate timetables without any clashes at all, and instead aimed to minimise
the number of clashes.
The best solution found had 83 students with at least one clash on their
timetable, or 4.59\% of the students.

\subsection{Binary Integer Programming}

Binary integer programming is used to solve constraint satisfiability problems.
Variables must either take a value of 1 or 0 (hence binary integer), as they 
are used to represent decisions, i.e., in the case of the university 
timetabling problem, a teaching session is happening in a specific room, at a 
specific time, on a specific day, with a specific teacher, with a specific 
teacher, with a specific student group, about a specific module, or 
not~\cite{bip_example}.
Then constraints are applied and used with the objective function to find what
value each variable takes.

First, a mathematical model of the problem must be constructed.
Abdellahi and Eledum (2006)~\cite{bip_example} modelled the features of the
university timetabling problem as a group of sets:
\begin{itemize}
	\item
		\begin{math}
			I = \{ 1, \dots , n_i \}
		\end{math}: set of days in the week where courses are offered
	\item
		\begin{math}
			J = \{ 1, \dots , n_j \}
		\end{math}: set of time slots in a day
	\item 
		\begin{math}
			K = \{ 1, \dots , n_k \}
		\end{math}: set of courses
	\item 
		\begin{math}
			L = \{ 1, \dots , n_l \}
		\end{math}: set of student groups
	\item 
		\begin{math}
			M = \{ 1, \dots , n_m \}
		\end{math}: set of teachers
	\item 
		\begin{math}
			N = \{ 1, \dots , n_n \}
		\end{math}: set of classrooms
\end{itemize}

Next, the decision variables are defined. The basic variables
\begin{math}
	x_{i,j,k,l,m,n}
\end{math}
are defined as
\begin{math}
	\forall i \in I, \forall j \in J, \forall k \in K, \forall l \in L, \forall
	m \in M, \forall n \in N
\end{math}
\begin{align*}
	x_{i,j,k,l,m,n} = 
	\begin{cases}
		1, & \parbox[t]{9.5cm}{if a course \( k \) taught by teacher \( m \) for
		the group of students \( l \) is assigned to the \( j^{th} \) time slot 
		of day \( i \) in classroom \( m \)} \\
		0, & \text{otherwise}
	\end{cases}
\end{align*}

In other words, the basic variables represent the decision of whether or a not
a course is being taught by a specific teacher for a specific group of students
is happening in a specific time slot of a specific day in a specific classroom.
Then the authors defined their auxiliary variables as

\begin{math}
	\forall i \in I, \forall j \in J, \forall l \in L
\end{math}
\begin{align*}
	y_{i,j,k,l} = 
	\begin{cases}
		1, & \parbox[t]{9.5cm}{if a course \( k + s \) for group of students
		overlap with its prerequite \( k \) for the same group \( l \) in the 
		\( j^{th} \) time slot of day \( i \)} \\
		0, & \text{otherwise}
	\end{cases}	
\end{align*}
\begin{math}
	\text{for } s \in 1, \ldots, n_k - 1, \text{where } k + s \leq n_k	
\end{math}

Finally, (people) defined two further sets of variables:
\begin{itemize}
	\item \( z_{im} \), which represents the existence of lectures for teacher 
		\( m \) on day \( i \)
	\item \( z_{il} \), which represents the existence of lectures for student 
		\( l \) on day \( i \)
\end{itemize}
\begin{math}
	\forall m \in M
\end{math}
\begin{align*}
	z_{im} =
	\begin{cases}
		1, & \text{if \( \sum_{j \in J} x_{i,j,k,l,m,n} \neq 0 \)} \\
		0, & \text{otherwise}
	\end{cases}	
\end{align*}
\begin{math}
	\forall l \in L
\end{math}
\begin{align*}
	z_{il} =
	\begin{cases}
		1, & \text{if \( \sum_{j \in J} x_{i,j,k,l,m,n} \neq 0 \)} \\
		0, & \text{otherwise}
	\end{cases}	
\end{align*}

\( z_{im} \) and \( z_{il} \) are for use with the object function, later. The
next thing to be modelled is the restraints.
For the university modelled by the authors, these were:

\begin{enumerate}
	\item There is no overlapping for courses
	\item Each teacher cannot be assigned to more than one course for any given
		period
	\item Each classroom cannot hold more than one course for any given period
	\item A student has some courses amount to 18 hours per week. Each course
		consists of 3 hours and taught in 2 periods (each period is 90 minutes
		long), which is expressed as the number of slots worked per day.
	\item For a student, each course occupies only one slot per day
	\item The lectures of each course must be distributed in such a way that
		there is one day off between them
	\item All lectures of a given course in a week must be held in the same
		classroom
	\item The overlap of course with prerequisites for the same group of
		undergrad students l is permitted (note: the authors refer to undergrad 
		students as pre-graduated students)
\end{enumerate}

As an example, the first restraint is modelled as:
\begin{align*}
	\sum_{k \in K} \sum_{m \in M} \sum_{n \in N} x_{i,j,k,l,m,n} \leq 1 \quad
	\forall i \in I, \forall j \in J, \forall l \in L
\end{align*}
The objective function needs to be either minimised or maximised (dependent on
how it is modelled), by changing the values assigned to the variables, whilst
ensuring that the constraints are met~\cite{objective_function}.
In this instance, the total dissatisfaction of teachers and students needs to be
minimised, which is equivalent to maximising the number of lectures per day,
which implies decreasing the waiting time between lectures.
\begin{align*}
	\parbox[t]{10cm}{(Total dissatisfaction) = Teacher\{number of lectures per
	day\} + Regular student\{number of lectures per day\} + Predicted graduate
	student\{number of lectures per day\}}
\end{align*}
The model produced may not be tractable, meaning that the problem may not be
able to be solved in a reasonable period.
To make the problem easier to be solved, the model needs to be reduced.
Abdellahi and Eledum achieved this by removing the index representing
classrooms (\( n\in N \)) and by adding another constraint, that the number of
courses cannot exceed the number of classrooms in a timeslot \( j \) in a given
day \( i \) (constraint 9).
To further simplify the model, the term corresponding to the overlapping of
courses and their prerequisites has been removed from the objective function,
with the related constraint 8.

The model is now:
\begin{equation*}
	\max \{1.5 \sum_{j \in J} \sum_{k \in K} \sum_{l \in L} \sum_{m \in M}
	x_{i,j,k,l,m} + \sum_{m \in M} z_{im} + \sum_{l \in L}z_{il} \}
\end{equation*}
\text{\rightline{(objective function)}}
subject to
\begin{equation*}
	\sum_{k \in K} \sum_{m \in M} x_{i,j,k,l,m} \leq 1 \quad \forall i \in I
	\forall j \in J, \forall l \in L
\end{equation*}
\text{\rightline{(Constraint 1)}}
\begin{equation*}
	\sum_{l \in L} \sum_{k \in K} x_{i,j,k,l,m} \leq 1 \quad \forall i \in I,
	\forall j \in J, \forall m \in M
\end{equation*}
\text{\rightline{(Constraint 2)}}
\begin{equation*}
	\sum_{j \in J} \sum_{k \in K} \sum_{m \in M} x_{i,j,k,l,m} \leq n_j \quad 
	\forall i \in I, \forall l \in L
\end{equation*}
\text{\rightline{(Constraint 3)}}
\begin{equation*}
	\sum_{j \in J}x_{i,j,k,l,m} \leq 1 \quad \forall i \in I, \forall k \in K,
	\forall l \in L, \forall m \in M
\end{equation*}
\text{\rightline{(Constraint 4)}}
\begin{equation*}
	\sum_{j \in J}(x_{i,j,k,l,m} + x_{i+2,j,k,l,m} = 2), i + 2 \leq 5 \quad
	\forall i \in I, \forall k \in K, \forall l \in L, \forall m \in M
\end{equation*}
\text{\rightline{(Constraint 5)}}
\begin{equation*}
	\sum_{k \in K} \sum_{l \in L} \sum_{m \in M} x_{i,j,k,l,m} \leq n_n \quad
	\forall i \in I, \forall j \in J 
\end{equation*}
\text{\rightline{(Constraint 9)}}
\begin{equation*}
	s \in \{1, \ldots, n_k - 1\}, k + s \leq n_k
\end{equation*}
\begin{equation*}
	\sum_{m \in M}z_{im} \leq 1 \quad \forall i \in I
\end{equation*}
\begin{equation*}
	\sum_{l \in L} z_{il} \leq n_l \quad \forall i \in I
\end{equation*}
\begin{equation*}
	x_{i,j,k,l,m}, y_{i,j,k,l,m}, z_{im}, z_{il} \leq 0 \quad \text{(so all
	variables are non-negative)}
\end{equation*}

The model can be reduced further into two models, one for students and one for
teachers.
The problem is solved with using an external penalty function method. Penalty
functions convert constrained problems to those without constraints by
introducing a penalty for violating the constraints~(\cite{penalty_function}).
By using a penalty function, the authors found a real solution
to the unrestrained problem, and then approximated it to a binary solution with
an algorithm.

When demonstrating their model, the authors assigned very small numbers to their
parameters: 5 workable days per week, 2 slots per day, 3 courses to be assigned,
2 student groups and 2 classrooms. A timetable was generated.

\subsection{Tabu Search}

Tabu search is a metaheuristic that guides a local search to explore the
solution space outside of the local optimum, by using a \textbf{Tabu list}.
A Tabu list is flexible memory structure that stores solutions that are not to
be used.
Tabu search is used to solve combinatorial optimisation problems, which have a
finite solution set.
The university timetabling problem has a finite solution set, as there are a
finite number of teachers, student groups, time slots, rooms, courses, etc., so
there is a finite number of different ways that the timetable can be created.
Tabu search makes uses of three main strategies:
\begin{itemize}
	\item Forbidding strategy: this controls what enters the Tabu list
	\item Freeing strategy: this controls what exits the Tabu list
	\item Short-term strategy: this is for managing interplay between the
		forbidding strategy and the freeing strategy to select trial solutions
\end{itemize}
Tabu search examines neighbouring solutions, which are solutions that are one
“step” away from the current solution.
Using the travelling salesman problem as an example, suppose the current
solution is B C D A F E, where each letter represents a city.
An example neighbouring solution is E C D A F B, where only one swap has
occurred, between the positions of B and E.
However, the solution E C D F A B is not in the neighbourhood of the current
solution, as two swaps have occurred.
If a solution has been used, then it is put into the Tabu list, until it meets
an \textbf{aspiration criterion}.
Aspiration criteria provide reasons for a solution to be freed from the Tabu
table (freeing strategy).
For example, if a using a Tabu move results in a solution better than any other
so far, then it can come out of the Tabu table. Much like with genetic
algorithms, a fitness function is used to measure how optimal a solution 
is~\cite{tabu_video}.

A basic algorithm is:
\begin{enumerate}
	\item Choose an initial solution \( i \) in the solution space \( S \).
		Set \( i^\prime=i \) and \( k = 0 \), where \( k \) is the number of
		iterations.
	\item Set \( k = k + 1 \) and generate a set of possible solutions
		\( N(i,k) \), where \( N(i,k) \) is the set of neighbouring solutions.
	\item Choose a best \( j \in N(i,k) \), where \( j \) is not Tabu (in the
		Tabu list).
		If \( j \) is Tabu, but meets an aspiration criterion, then choose
		\( j \).
		Set \( i = j \).
	\item If \( f(i) > f(i^\prime) \) (where \( f \) is the fitness function),
		set \( i^\prime = i \).
		Note this is for the case when a higher fitness function is better.
		For the case when a lower fitness function is better, set
		\( i^\prime = i \) when \( f(i) < f(i^\prime) \).
	\item Put \( i \) into the Tabu table. Update aspiration criteria.
	\item If a stopping condition is met, stop. Else, go to step 2.
\end{enumerate}
There are several possible stopping criteria, such as:
\begin{itemize}
	\item There are no more feasible solutions in the neighbourhood of the
		current solution
	\item \( k \) is larger than the maximum number of allowed iterations
	\item The number of iterations since the last improvement of \( i^\prime \)
		is greater than a specified number –-- meaning that convergence has been
		reached, and that there are diminishing returns on finding a more
		optimal solution
	\item An optimal solution has been obtained.
		There would need to be a method to find what the vicinity of the optimal
		solution, so that upper lower bounds can be set.
\end{itemize}
Tabu search has been used to solve the university timetabling
problem~\cite{tabu_example}.
Awad et al.\ defined four different neighbourhoods:
\begin{itemize}
	\item Nb1: Randomly choose a particular course and progress to a feasible
		timeslot, which can produce the smallest cost
	\item Nb2: Randomly select a room. Also, randomly select two courses for
		that room. Next, swap timeslots.
	\item Nb3: Randomly select two times. Next, swap timeslots.
	\item Nb4: Randomly select a particular time and swap it with another time,
		in the range between 0 and 44, which can produce the smallest penalty
		cost.
\end{itemize}
In order to produce an initial solution which met all the hard constraints, a
least saturation degree algorithm is used, where events that are more different
to schedule are scheduled first.
If that did not produce a feasible solution, then Nb1 is used for a specific
number of repetitions, then Nb2 is used for a specific number of repetitions if
Nb1 did not reach a feasible solution.
Next an improvement algorithm was used with Nb3 and Nb4, specifically adaptive
Tabu search.
The penalty cost is checked every 1,000 iterations and if the penalty cost has
not changed, then two solutions are removed from the Tabu List.

The authors compared their algorithm to the work of others, by running 11
different datasets through them, and totalling the number of violations of their
hard and soft constraints. 
The authors' algorithm performed best against other implementations of Tabu 
search, and was also compared to other algorithms that did not use Tabu search 
at all.
The authors' implementation performed better than all the other algorithms, 
except variable neighbourhood search on the largest dataset, and simulated 
annealing which performed better across all dataset sizes.

\newpage

\subsection{Answer Set Programming}

Answer set programming (ASP) reduces problems to logic programs, and then
answer set solvers are used to do the search~\cite{asp_example}. 
The logic programs are sets of rules that take the form:
\begin{center}
	\( a_0 \) \verb|:-| \( a_1, \ldots, a_m \) \verb|not| \( a_{m+1}, \ldots, \)
	\verb|not| \( a_n \)
\end{center}
where every \( a_i \) is a propositional atom and \verb|not| is default 
negation.
If \( n = 0 \) then a rule is a fact. 
A rule is an integrity constraint if \( a_0 \) is omitted.
A logic program induces a collection of answer sets, which are recognized models
of the program determined by answer set semantics.

To make ASP better for real-world use, some extensions were developed.
Rules with first-order variables are viewed as shorthand for the set of their
ground instances.
Additionally, there are conditional literals that take the form:
\begin{center}
	\verb|a: | \( b_1, \ldots, b_m \)
\end{center}
where \verb|a| and \( b_i \) are possibly default-negated variables.
There are also cardinality constraints which take the form:
\begin{center}
	\verb|s| \( \{c_1, \ldots, c_n\} \) \verb|t|
\end{center}
where each \( c_j \) is a conditional literal.
\verb|s| and \verb|t| provide lower and upper bounds on the number of satisfied
literals in the cardinality constraint.
For example, \verb|2{a(X):b(X)}4| is true when 2, 3 or 4 instances of \verb|(X)|
(subject to \verb|b(X)|) are true. 
Also \( N=c_1,\ldots c_n \) binds \( N \) to the number of satisfied conditional
literals \( c_j \).
And objective functions that minimise the sum of weights \( w_j \) of
conditional literals \( c_j \) are expressed as \verb|#minimize{| \( w_1:c_1,
\ldots, w_n:c_n \) \verb|}|.

The logic programs can be written in a language called AnsProlog (Answer Set
Programming in Logic), which can be used with answer set solvers such as
smodels.

Banbara et al. (2019)\cite{asp_example} used Answer Set Programming to solve the 
timetabling problem.
They split their constraints into hard and soft constraints, where the soft
constraints have either constant cost or calculated cost.
If a constraint has constant cost, then there is one penalty point per
violation, whereas constraints that have a calculated cost attached to them
have their penalty points calculated dynamically with each violation.
The authors defined their constraints as:
\begin{itemize}
	\item \( H_1 \) \textbf{Lectures}: All lectures of each course must be 
		scheduled, and they must be assigned to distinct timeslots.
	\item \( H_2 \) \textbf{Conflicts}: Lectures of courses in the same 
		curriculum or taught by the same teacher must be all scheduled in 
		different timeslots.
	\item \( H_3 \) \textbf{RoomOccupancy}: Two lectures cannot take place in 
		in the same room in the same timeslot.
	\item \( H_4 \) \textbf{Availability}: If the teacher of the course is 
		unavailable to teach that course at a given timeslot, then no lecture of
		the course can be scheduled at that timeslot.
	\item \( S_1 \) \textbf{RoomCapacity}: For each lecture, the number of
		students that attend the course must be less than or equal the number of
		seats of all the room that host its lectures.
		The penalty points, reflecting the number of students above the
		capacity, are imposed on each violation.
	\item \( S_2 \) \textbf{MinWorkingDays}: The lectures of each course must be
		spread into a given minimum number of days.
		The penalty points, reflecting the number of days below the minimum, are
		imposed on each violation.
	\item \( S_3 \) \textbf{IsolatedLectures}: Lectures belonging to a
		curriculum should be adjacent to each other in consecutive timeslots.
		For a given curriculum we account for a violation every time there is 
		one lecture not adjacent to any other lecture within the same day.
		Each isolated lecture in a curriculum counts as one violation.
	\item \( S_4 \) \textbf{Windows}: Lectures belonging to a curriculum should 
		not have time windows (periods without teaching) between them.
		For a given curriculum we account for a violation every time there is
		one window between two lectures within the same day.
		The penalty points, reflecting the length in periods of time window, are
		imposed on each violation.
	\item \( S_5 \) \textbf{RoomStability}: All lectures of a course should be
		given in the same room.
		The penalty points, reflecting the number of distinct rooms but the
		first, are imposed on each violation.
	\item \( S_6 \) \textbf{StudentMinMaxLoad}: For each curriculum, the number
		of daily lectures should be within a given range.
		The penalty points, reflecting the number of lectures below the minimum
		or above the maximum, are imposed on each violation.
	\item \( S_7 \) \textbf{TravelDistance}: Students should have the time to
		move from one building to another one between two lectures.
		For a given curriculum we account for a violation every time there is an
		instantaneous move: two lectures in rooms located in different buildings
		in two adjacent periods within the same day.
		Each instantaneous move in a curriculum counts as 1 violation.
	\item \( S_8 \) \textbf{RoomSuitability}: Some rooms may not be suitable for
		a given course because of the absence of necessary equipment.
		Each lecture of a course in an unsuitable room counts as 1 violation.
	\item \( S_9 \) \textbf{DoubleLectures}: Some courses require that lectures
		in the same day are grouped together (double lectures).
		For a course that requires grouped lectures, every time there is more
		than one lecture in one day, a lecture non-grouped to another is not
		allowed.
		Two lectures are grouped if they are adjacent and in the same room.
		Each non-grouped lecture counts as 1 violation.
\end{itemize}
A solution is feasible when all the hard constraints are satisfied, but the
objective is to find a solution with the minimal penalty cost.
The authors formulated the problem in different ways, where a formulation is
defined as specific set of soft constraints together with weights associated
with each of them.
Different formulations allow for timetables to be generated for different
scenarios --- for instance, double lectures may not be required, so the 
constraint \( S_9 \) would not be used.
Formally, the timetabling problem is formulated as a combinatorial optimisation
problem with the objective function to minimise the weighted sum of the penalty
points.

Next, the authors encoded the constraints and facts into the correct format 
for the specific ASP system, which then returned an assignment representing a
solution.

When testing, the authors used multiple sets of input data, some of which were 
very large, for instance, one of them consisted of 850 courses, 132 rooms, 850 
curricula, 7,780 unavailability constraints, and 45,603 room constraints.
The system was run on a cluster of machines with server-grade hardware, which 
was probably necessary given the size of the datasets.
They also used different configurations for their ASP system, all of which 
managed to produce optimal solutions.

\subsection{Bat Inspired Algorithm}

The Bat Inspired Algorithm (BA) is a heuristic method originally proposed by 
Yang (2010)~\cite{yang_bat}.
It is inspired by echolocation, which is a method used by bats and other animals
to navigate, using sound.
To simplify echolocation, Yang suggested these rules:
\begin{enumerate}
	\item Bats use echolocation to determine distance, and they can
		differentiate between food and barriers.
	\item Bats fly randomly with velocity \( v_i \) at position \( x_i \) having
		a fixed frequency \( f_{\min} \), varying wavelength \( \lambda \) and 
		loudness \( A_i \) to search for prey.
		In this rule, it is assumed that bats can adjust automatically the
		frequency (or wavelength) of their emitted pulses as wall the rate of
		pulse emission \( r \in [0,1] \).
		The automatic adjustment relies on the proximity of the target.
	\item The loudness of the bats can vary in many ways, however, it is assumed
		that the loudness can vary from between positive values \( A_0 \) and 
		\( A_{\min} \).
\end{enumerate}
Pseudocode for the Bat Algorithm can be seen in 
listing~\ref*{listing:bat-algorithm}.

\begin{listing}[!ht]
	\inputminted[linenos, fontsize=\footnotesize]{text}{code/bat-algorithm.txt}
	\caption{Pseudocode for the bat algorithm~\cite{ba_example}}
	\label{listing:bat-algorithm}
\end{listing}

In the first step, the bat population initialised with parameters position
\( x_i \), velocity \( v_i \) and frequency \( f_i \).
With each generation, every bat moves by changing velocity \( v_i^t \) and 
position \( x_i^t \) at time \( t \) with the equations:
\begin{equation}
	f_i = f_{\min} + (f_{\max} - f_{\min}) \beta
\end{equation}
\begin{equation}
	v_i^t = v_i^{t-1} + (x_i^{t-1} - x_*) f_i
\end{equation}
\begin{equation}
	x_i^t = x_i^{t-1} + v_i^t
\end{equation}
where \( \beta \) is a random number from the interval \( [0,1] \) and \( x_* \)
represents the current best global solution among all the bats in the
population.

In the local search part, a new solution for each bat is generated using a
random walk:
\begin{equation*}
	x_{new} = x_{old} + \epsilon < A^t >
\end{equation*}
where \( \epsilon \) is a random number from the interval \( [0,1] \) and
\( < A^t > \) is the average loudness of the bats in the population at a
specific time step \( t \).
Also the rate of sound pulse emission \( r_i \) and the loudness \( A_i \) of 
each bat needs to updated with the equation:
\begin{equation*}
	A_i^t = \alpha A_i^{t-1}
\end{equation*}
\begin{equation*}
	r_i^t = r_i^0 [1 - e^{-\gamma t}]
\end{equation*}
where \( \alpha \) and \( \gamma \) are constants.

The above BA was designed for continuous optimisation problems, but as the
timetabling problem is a discrete combinatorial optimisation problem, Limota et
al. (2021)~\cite{ba_example} modified the algorithm.
In the context of the timetabling problem, \( i \) represents a candidate 
solution and each position \( x_i \) is initialised by randomly choosing a
lecture and then an algorithm tries to find timeslots without clashes and a 
large enough room. This is to start with a good initial timetable.

The parameter frequency \( f_i \) is fixed to \( 1 \) to reduce the complexity 
of the algorithm, meaning that velocity is now calculated as:
\begin{equation*}
	v_i^t = f(x_i) - f(x_*)
\end{equation*}
\( v_i \) can be interpreted as the number of operations that bat \( i \) will 
perform in updating the current position. The new position \( x_i^t \) is
calculated with Equation 1.3, which implies that the current position is
dependent on velocity and the previous position. But now, this can be 
interpreted to mean that the current position is obtained by performing 
\( v_i^t \) moves.

To update the current solution of an instance of the timetabling problem, a
lecture is moved from one timeslot to another randomly selected one.

When testing their algorithm, the authors used input data consisting of 
124 courses, 273 lectures, 10 rooms and 5551 students. They ran their algorithm 
4 times, and each time generated timetables where all the hard constraints were
satisfied.

\newpage

\section{Comparison of Methods}

Table~\ref*{table:method-comparison} summarises how the different methods for 
solving the timetabling problem performed.

\begin{table}[h]
	\begin{tabular}{cccc}
		\toprule
		Method 
			& \multirow{2}{9em}{Dataset size (number of courses)}
			& \multirow{2}{6em}{Solution generated?}
			& Errors? \\
		\\
		\midrule
		\multirow{2}{10em}{Genetic Algorithm\cite{ga_example}}
			& 340
			& Yes 
			& \multirow{2}{9em}{84 students had at least one clash} \\
		\\
		\multirow{2}{10em}{Binary Integer Programming\cite{bip_example}}
			& 3
			& Yes
			& None \\
		\\
		\multirow{1}{10em}{Tabu Search\cite{tabu_example}}
			& 400*
			& Yes 
			& \multirow{2}{9em}{347* hard/soft constraint violations} \\
		\\
		\multirow{2}{10em}{Answer Set Programming\cite{asp_example}}
			& 850
			& Yes 
			& None \\
		\\
		\multirow{2}{10em}{Bat Inspired Algorithm\cite{ba_example}}
			& 124
			& Yes
			& None \\
		\\
		\bottomrule
	\end{tabular}
	\caption{Summative table of the results of the different methods}
	\label{table:method-comparison}
\end{table}
*largest dataset

It is also worth recalling a few other details about the different methods used:
\begin{itemize}
	\item At the university used for the genetic algorithm 
		method~\cite{ga_example}, courses are arranged in such a way that 
		essentially every student has a unique timetable, therefore making the 
		problem more complex compared to if students were put into groups.
	\item Multiple different configurations were used for the answer set 
		programming system~\cite{asp_example}, some of which were able to 
		produce a greater number of optimal solutions than others.
\end{itemize}

\newpage

\section{Chosen Method}

I will not be using binary integer programming as my chosen method, as it was 
only tested on such a small dataset it is unknown how performance will scale for
larger datasets.
Likewise, I will not be using answer set programming as the dataset was so large
that the algorithm had to be run on server-grade hardware, which I do not have 
access to, and I do not know how the performance scales with a decrease in the 
dataset size.
I am not going to use Tabu search, as there were a large number of constraint 
violations in relation to the size of the dataset.
I have hence decided to use a genetic algorithm as I think it would be easier to
implement than a bat inspired algorithm.