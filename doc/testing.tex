All the testing was run on computer using Arch Linux x86\textunderscore64, with 
a dual core processor clocked @ 2.7 GHz, and 8 GB of RAM.
I used 3 different sizes of data sets, and used different sizes of population, 
but kept the chance of mutation constant at \( \frac{1}{1000} \).
I used the number of generations to generate a correct solution to measure the
speed of the program with the different inputs.
Table~\ref*{table:results-mean} shows the mean number of generations for each
number of modules, for each population size.
The full results are in table~\ref*{table:results-full}.

\begin{table}
	\begin{tabular}{c|ccc}
		\toprule
		& \multicolumn{3}{c}{Number of Modules} \\
		Population Size & 10 & 50 & 100 \\
		\midrule
		10 & 155.4 \\
		50 & 16.7 \\
		100 & 11 \\
		200 & 8.8 \\
		\bottomrule
	\end{tabular}
	\caption{Mean average of number of generations for each number of modules
		and population size}
	\label{table:results-mean}
\end{table}

\begin{table}
	\begin{tabular}{cc|cccc|cccc|cccc}
		\toprule
		\multicolumn{2}{c}{Num of Modules}
			& \multicolumn{4}{c}{10}
			& \multicolumn{4}{c}{50}
			& \multicolumn{4}{c}{100} \\
		\midrule
		\multicolumn{2}{c}{Population Size}
			& 10 & 50 & 100 & 200 
			& 10 & 50 & 100 & 200 
			& 10 & 50 & 100 & 200 \\
		\midrule
		Result & 1
			& 68 & 14 & 9 & 7
		\\
		& 2
			& 86 & 17 & 9 & 9
		\\
		& 3
			& 91 & 7 & 12 & 8
		\\
		& 4
			& 117 & 10 & 11 & 8
		\\
		& 5
			& 265 & 9 & 10 & 10
		\\
		& 6
			& 157 & 24 & 13 & 12
		\\
		& 7
			& 172 & 32 & 17 & 10
		\\
		& 8
			& 186 & 14 & 8 & 7
		\\
		& 9
			& 188 & 177 & 10 & 9
		\\
		& 10
			& 224 & 23 & 11 & 8
		\\
		\midrule
		\multicolumn{2}{c}{Mean}
			& 155.4 & 16.7 & 11 & 8.8
		\\
		\multicolumn{2}{c}{Std Deviation}
			& 64.2 & 7.7 & 2.6 & 1.6
		\\
		\bottomrule
	\end{tabular}
	\caption{Full testing results}
	\label{table:results-full}
\end{table}

