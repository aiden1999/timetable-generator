\section{Fulfilling Requirements}

I managed to achieve all the necessary requirements listed in \S2.1, except for
requirements 2(e) (a session is in the correct type of room, for example, a
lecture is in a lecture theatre, and a lab session is in a lab), 2(f) (the 
number of students in the room must not exceed the capacity of the room), and
2(g) (the size of the room is optimal for the number of students, meaning that
the size of the room is as small as possible).
This was because I ran out of time.

Of the aspirational requirements in \S2.2, I implemented 2 - allow a user to
change the parameters of the genetic algorithm.

The program does work for the most part, but when looking through a solution for
100 modules, I did spot the occasional clash, but this may have been because
the problem was unsolvable (this will be elaborated on later).

\section{Comparison to the Work of Others}

Comparing to the other methods explored earlier and summarised in 
table~\ref*{table:method-comparison}, my solution did not turn out to be the
most effective, considering the issue listed above.

\section{Other Improvements}

In addition to the other aspirational requirements, there are other things I 
would do if I were to do this again.

Firstly, to address the issue as to whether or not the problem is solvable for
the input, I would change the program so that it would check if the problem is
solvable in the first place. 
As a simple example, the problem would not be solvable if a teacher was assigned
to more sessions than as many as they had listed as preferred time slots.
If it is not possible for a correct timetable to be generated, the user would
be informed of the reason.
Similarly, the program would perform data validation checking on the input data
before attempting to generate a solution, for example, checking if a module does
not have 0 for both the number of lecture hours and the number of lab hours. 

To further elaborate on the aspirational requirements, the use of graphical 
interface would also allow the user to input the data in a way that would be
more familiar to them, as opposed to editing JSON files.
This could also be where any problem with the input data, such as listed above,
would be highlighted on the fly, so the user could change it straight away.
The final timetables could perhaps be returned in a file format of the user's
choosing, for instance a CSV file, or as an HTML snippet for use on a website.

Finally, I would use an object oriented programming approach, so I could model
each session as a object, with time slot, teacher, student group, module and 
room as attributes.
This would make accessing the different parts of a session easier.
Solutions would be objects made of a list of sessions, and the population would
be an object which is a list of sessions.