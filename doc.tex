\documentclass[a4paper, 12pt]{report}

\usepackage[dvips]{graphicx}
\usepackage{listings}
\usepackage{color}
\usepackage{url}
\usepackage{fancybox}
\usepackage{parskip}

% Choose your bibliography style
%\bibliographystyle{}

\usepackage[includeheadfoot,margin=3.5cm]{geometry}
\usepackage{fancyhdr}
\usepackage{titlesec}
\usepackage{amsmath}

\linespread{1.3}

% Setup headers and footers
\fancyhf{}

% Page number placed on right side on odd pages and left side on even pages
\fancyfoot[RO, LE] {\thepage}

\begin{document}

% Front Page

\thispagestyle{empty}

\fancypage{}{\fbox}

\begin{center}
	\Large{
		\hfill \begin{tabular}{l}
			Computer Science \\ and Mathematics \\
			COC255           \\
			B827126
		\end{tabular}
	}

	\vspace*{\fill}

	\Large{\textbf{TIMETABLE CREATION \\ USING \\
			ARTIFICIAL INTELLIGENCE}}
	\vspace*{\fill}

	by
	\vspace*{\fill}

	Aiden Nico Tempest

	\vspace*{\fill}
	Supervisor: Dr.\ S. Fatima
	\vspace*{\fill}

	\underline{Department of Computer Science} \\
	\underline{Loughborough University}
	\vspace*{\fill}
	
	August 2023
\end{center} % Front Page ends

\newpage

\fancypage{}{}  %Reset so that next pages do not have a box around them
\tableofcontents

\newpage

\chapter{Literature Review}

\section{The Timetabling Problem}

The timetabling problem is a constraint satisfying problem and it is known to
be NP-Complete. The aim is to create a university timetable where several
constraints are met. These constraints are either hard constraints or soft
constraints. For a solution to be valid, all the hard constraints must be met.
Not all (or in fact, any) of the soft constraints need to be met for the
solution to be valid, however it is preferable for as many soft constraints to
be met as possible. Examples of possible hard constraints include:

\begin{itemize}
	\item At most only one session (i.e., a lecture or lab) is happening in a 
		specific room in a specific period
	\item A student can only be attending at most one session in a specific 
		period
	\item A teacher (e.g., a lecturer or lab helper) can only be attending at 
		most one session in a specific period
	\item The size of a student group cannot exceed the capacity of the room
	\item The room must be appropriate for the type of session, e.g., a lecture 
		must be in a lecture theatre, and a lab session must be in a computer 
		lab
	\item Part time teachers can only be assigned certain time slots, e.g., they may not
	      work on Tuesdays, so sessions they teach cannot be scheduled for time slots on
	      Tuesday
\end{itemize}

Possible soft constraints include:

\begin{itemize}
	\item Students do not have more than two consecutive hours scheduled
	\item The capacity of a room is well suited to the size of the student group, to make
	      an efficient use of space e.g., a group of twenty students are not going to be
	      in a room with a capacity of two hundred
	\item If a student or teacher has one session immediately after another, then the
	      respective rooms are relatively close to each other
\end{itemize}
A solution is invalid if at least of one the hard constraints are met. For 
example, if a student is scheduled to be in two different sessions at the same 
time - this is known as a clash.

\section{Methods}

\subsection{Genetic Algorithms}

Genetic algorithms (GAs) made up a group of search metaheuristics, inspired by
Darwin's theory of evolution. Here, the fittest members of a population survive
and produce offspring, which inherit the characteristics of the parents. It is
also possible for the offspring to have small mutations within their genetic
code, which may or may not be beneficial towards the population's survival.
This theory can be applied to search problems. In this case, the population
represents the search space, which is a collection of candidate solutions to a
problem, and the population of solutions evolves as the algorithm searches for
a desired solution.

There does not exist a rigorous definition of GAs, but most methods use these
five phases:

\begin{enumerate}
	\item Initial population - populations of chromosomes
	\item Fitness function
	\item Selection - according to fitness
	\item Crossover - to produce new offspring
	\item Mutation - random mutation of new offspring
\end{enumerate}

The initial \textbf{population} is a set of \textbf{individuals}, where each
individual represents a candidate solution to the problem. These solutions will
almost definitely not satisfy the problem, as they are randomly generated, but
that does not matter.

An individual (and hence that candidate solution), is defined by its
\textbf{chromosome}. The chromosome is often encoded as a string of binary
characters; however, any alphabet can be used. These characters are called
\textbf{alleles}, and a single character or group of adjacent characters that
encode a particular element of the candidate solution are known as a
\textbf{gene}.

Using the example of the university timetabling problem, several alleles may be
used to encode a room, but together they are one gene.

The \textbf{fitness} function measures how well a candidate solution solves the
problem. For example, in the instance of the university timetabling problem,
the fitness of the solution could be measured by the number of times in the
solution that a hard constraint is not met. This means that a solution with a
score of 0 is a valid solution, as there are instances of a hard constraint not
being met.

Once the fitness function has been used to calculate the fitness of each
individual, the fittest individuals are chosen for reproduction in the
\textbf{selection} phase. There are several ways to select individuals to use
for producing offspring. One way is to simply choose the two individuals with
the best fitness. Another way is to use roulette-wheel sampling, where any
individual could be chosen for producing offspring, but fitter individuals are
more likely to be chosen. This second method introduces more variation into the
offspring, to reduce the chance of convergence onto a local maximum.

The \textbf{crossover} phase, or reproduction phase, is where genetic material
is exchanged between two parents. The crossover operator randomly chooses a
locus (position) in a chromosome, in between alleles. The subsequent before and
after parts of the chromosome are swapped between the two parents to produce
two offspring. This is repeated multiple times to produce a population of the
same size as the initial population. Whether this is by two parents reproducing
multiple times using different loci or not, depends on the method used in the
selection phase.

After reproduction, \textbf{mutations} can be introduced into the offspring. 
For example, if the chromosomes are encoded by binary strings, then some bits 
are randomly flipped. However, there is a very small chance of this occurring 
at each bit, a suggested probability is 0.1\%. By introducing mutations into 
the offspring, the likelihood of reaching a local maximum is reduced.

Once there is a new population made up of the offspring, the process repeats 
until a solution is found. Each repetition is called a \textbf{generation}, and 
the entire set of generations is known as a \textbf{run}.

\subsection{Binary Integer Programming}

Binary integer programming is used to solve constraint satisfiability problems.
Variables must either take a value of 1 or 0 (hence binary integer), as they 
are used to represent decisions, i.e., in the case of the university 
timetabling problem, a teaching session is happening in a specific room, at a 
specific time, on a specific day, with a specific teacher, with a specific 
teacher, with a specific student group, about a specific module, or not. Then 
constraints are applied and used with the objective function to find what value 
each variables takes.

First, a mathematical model of the problem must be constructed. REF modelled 
the features of the university timetabling problem as a group of sets:
\begin{itemize}
	\item
		\begin{math}
			I = \{1, \dots , n_i\}
		\end{math}
		: set of days in the week where courses are offered
	\item
		\begin{math}
			J = \{1, \dots , n_j\}
		\end{math}
		: set of time slots in a day
	\item 
		\begin{math}
			K = \{1, \dots , n_k\}
		\end{math}
		: set of courses
	\item 
		\begin{math}
			L = \{1, \dots , n_l\}
		\end{math}
		: set of student groups
	\item 
		\begin{math}
			M = \{1, \dots , n_m\}
		\end{math}
		: set of teachers
	\item 
		\begin{math}
			N = \{1, \dots , n_n\}
		\end{math}
		: set of classrooms
\end{itemize}

Next, the decision variables are defined. The basic variables
\begin{math}
	x_{i,j,k,l,m,n}
\end{math}
are defined as \\
\begin{math}
	\forall i \in I, \forall j \in J, \forall k \in K, \forall l \in L, \forall
	m \in M, \forall n \in N
\end{math}
\begin{align}
	x_{i,j,k,l,m,n} = 
	\begin{cases}
		1, & \parbox[t]{9.5cm}{if a course $k$ taught by teacher $m$ for the
		group of students $l$ is assigned to the $j^{th}$ time slot of day $i$
		in classroom $m$} \\
		0, & \text{otherwise}
	\end{cases}	
\end{align}


In other words, the basic variables represent the decision of whether or a not
a course is being taught by a specific teacher for a specific group of students
is happening in a specific time slot of a specific day in a specific classroom
Then (Abdellahi and Eledum, 2017) defined their auxiliary variables as \\

\begin{math}
	\forall i \in I, \forall j \in J, \forall l \in L
\end{math}
\begin{align}
	y_{i,j,k,l} = 
	\begin{cases}
		1, & \parbox[t]{9.5cm}{if a course $k + s$ for group of students
		overlap with its prerequite $k$ for the same group $l$ in the $j^{th}$
		time slot of day $i$} \\
		0, & \text{otherwise}
	\end{cases}	
\end{align}
\begin{math}
	\text{for } s \in 1, \cdots, n_k - 1, \text{where } k + s \leq n_k	
\end{math}

Finally, (people) defined two further sets of variables:
\begin{itemize}
	\item $z_{im}$, which represents the existence of lectures for teacher $m$
	on day $i$
	\item $z_{il}$, which represents the existence of lectures for student $l$
	on day $i$
\end{itemize}

\begin{math}
	\forall m \in M
\end{math}
\begin{align}
	z_{im} =
	\begin{cases}
		1, & \text{if $\sum_{j \in J} x_{i,j,k,l,m,n} \neq 0$} \\
		0, & \text{otherwise}
	\end{cases}	
\end{align}
\begin{math}
	\forall l \in L
\end{math}
\begin{align}
	z_{il} =
	\begin{cases}
		1, & \text{if $\sum_{j \in J} x_{i,j,k,l,m,n} \neq 0$} \\
		0, & \text{otherwise}
	\end{cases}	
\end{align}

$z_{im}$ and $z_{il}$ are for use with the object function, later. The next
thing to be modelled is the restraints. For the university modelled by
(Abdellahi and Eledum, 2017), these were:

\begin{enumerate}
	\item There is no overlapping for courses
	\item Each teacher cannot be assigned to more than one course for any given
	period
	\item Each classroom cannot hold more than one course for any given period
	\item A student has some courses amount to 18 hours per week. Each course
	consists of 3 hours and taught in 2 periods (each period is 90 minutes
	long), which is expressed as the number of slots worked per day.
	\item For a student, each course occupies only one slot per day
	\item The lectures of each course must be distributed in such a way that
	there is one day off between them
	\item All lectures of a given course in a week must be held in the same
	classroom
	\item The overlap of course with prerequisites for the same group of
	undergrad students l is permitted (note: (Abdellahi and Eledum, 2017) refer
	to undergrad students as pre-graduated students)
\end{enumerate}

As an example, the first restraint is modelled as:

% Rename Bibliography to References
% \renewcommand\bibname{References}
%\bibliography{<Path to .bib file>}

% Include appendix section if needed
%\include{Appendix/Appendix}

\end{document}